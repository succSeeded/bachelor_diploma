% !TeX spellcheck = ru_RU-Russian
% !TeX encoding = UTF-8 

\documentclass[12pt,a4paper]{extarticle}

\usepackage[russian]{babel}
\usepackage[utf8]{inputenc}
\usepackage{setspace} 
\usepackage[a4paper,
		left=30mm,
		right=10mm,
		top=20mm,
		bottom=20mm]{geometry}
\usepackage{amsmath, amssymb}
\usepackage{amsthm}
\usepackage{graphicx, cite}
\usepackage{enumitem}
\usepackage{kprj} 

\NIR

\title{Метод выбора функции Ляпунова в моделях с дробно-рациональными правыми частями}
\author{М.Д.~Кирдин}
\authorfull{Кирдин Матвей Дмитриевич}
\group{ФН12-81Б}
\faculty{Фундаментальные науки}
\chair{Математическое моделирование}
\chief{Д.А.~Фетисов}
\workyear{2024}
\facultyshort{ФН}
\chairshort{ФН12}
\chairhead{А.П.~Крищенко}

\onehalfspacing
\setcounter{secnumdepth}{3}
\renewcommand{\qedsymbol}{$\blacktriangleright$}
\renewenvironment{proof}{\noindent$\blacktriangleleft$}{}
\theoremstyle{definition}
\newtheorem{theorem}{Теорема}
\theoremstyle{definition}
\newtheorem{definition}{Определение}
\theoremstyle{definition}
\newtheorem{affirmation}{Утверждение}
\setcontentsname{СОДЕРЖАНИЕ}
\setrefsname{СПИСОК ИСПОЛЬЗОВАННЫХ ИСТОЧНИКОВ}

\begin{document}
	\begin{spacing}{1.0}
		\maketitle
	\end{spacing}
	
	\tableofcontents
	 
	\section{Постановка задачи}
	
	Рассмотрим следующую пятимерную систему с положительными параметрами:
	\begin{equation}
		\begin{cases}
			\begin{aligned}
				\dot{x}_1 &= r_1x_1\left(1-\dfrac{x_1}{c_1}\right)-\dfrac{1}{x_4+e_1}(\alpha_1x_2+\alpha_2x_3)\dfrac{x_1}{x_1+k_1},\\
				\dot{x}_2 &= r_2x_2\left(1-\dfrac{x_2}{c_2}\right)+\dfrac{x_5}{k_4+x_5}a_1\dfrac{1}{x_4+e_2}-\alpha_3\dfrac{x_1}{x_1+k_2}x_2,\\
				\dot{x}_3 &= a_2\dfrac{x_1}{k_5+x_4}-\mu_1x_3-\alpha_4\dfrac{x_1}{x_1+k_3}x_3,\\
				\dot{x}_4 &= s_1 + b_1x_1-\mu_2x_4,\\
				\dot{x}_5 &= b_2x_3-\mu_3x_5,
			\end{aligned}
		\end{cases}\label{eq:initial_system}
	\end{equation}
	где $x=(x_1, x_2, x_3, x_4, x_5)$ -- неотрицательные переменные, $t\ge0$ --- время.
	
	Введем следующие обозначения:
	\[\mathbb{R}^n_{+,0}=\{x=(x_1,\dots,x_n)\in\mathbb{R}^n:\, x_i\ge0,\, i=\overline{1,n}\},\,\mathbb{R}_{+,0}=\{x\in\mathbb{R}:\, x\ge0\}.\]
	
	Для данной системы найдем положения равновесия при помощи построения функций Ляпунова, а также условия их устойчивости, если таковые имеются. 
	
	\section{Основная часть}
	
	Рассмотрим следующий класс динамических систем:
	\[\dot{x}_i=f_i(x),\, x=(x_1,\dots,x_n)\in\mathbb{R}^{n}_{+,0},\, i=\overline{1,n},\]
	где правые части $f_i(x)$ --- некие дробно-рациональные функции (далее ДРФ) вида 
	\begin{equation*}
		g(x)=\dfrac{P(x)}{Q(x)},\quad l,\, m\in\mathbb{N}\cap\{0\}.
	\end{equation*}
	Здесь $P(x)$ и $Q(x)$ --- многочлены порядков $l$ и $m$ соответственно с отрицательными действительными корнями. 
	
	\begin{theorem}
		Для динамических систем данного  вида за функцию Ляпунова для внутреннего положения равновесия $x_0=(x_{1,0},\dots,x_{n,0})\in\mathbb{R}^n_{+}$ можно принять следующее выражение:
		\[V(x)=2\sum\limits_{i\in\sigma_1}\hat{k}_i(x_i-x_{i,0}-x_{i,0}\ln\dfrac{x_i}{x_{i,0}})+\dfrac{1}{2}\sum\limits_{j\in\sigma_2}\hat{k}_j(x_j-x_{j,0})^2,\]
		где $\sigma_1$ -- множество номеров функций $f_i(x)$ кратных $x_i$;\\ 
		$\sigma_2$ -- множество всех остальных номеров функций $f_j(x)$;\\
		$\hat{k}_i$ -- положительные параметры. 
		
		Производная такой функции в силу системы будет представима в виде квадратичной формы:
		\[\dot{V}(x)=(x-x_0)^{T}H(x)(x-x_0),\]
		где $H(x)$ -- симметричная функциональная матрица размера $n\times n$, координатными функциями которой являются константы либо рациональные функции. 
	\end{theorem}
	\begin{proof}
		Обозначим корни $Q_i(x)$ как $(-a_k)\in\mathbb{R}_{-},\, k=\overline{1,m}$, а корни $P_i(x)$ как $(-b_j)\in\mathbb{R}_{-},\, j=\overline{1,l}$. Тогда
		\begin{align*}
			Q_i(x)=&(x_{k_1}+a_1)\cdot\dots\cdot(x_{k_m}+a_m),\\
			P_i(x)=&(x_{j_1}+b_1)\cdot\dots\cdot(x_{j_l}+b_l).
		\end{align*}
		Здесь $j_1,\dots,j_l,k_1,\dots,k_m\in\{1,\dots,n\}$. Заметим, что во внутреннем ПР $x=x_0$ справедливо, что:
		\[f_i(x_0)=\dfrac{P_i(x_0)}{Q_i(x_0)}=0.\]
		
		На области $\mathbb{R}^n_{+}$ для $V(x)$ выполняются следующие условия:
		\[V(x)>0,\, V(x_0)=0,\, x_0\in D,\,x\in D\setminus\{x_0\}.\]
		Квадратичные слагаемые неотрицательно определены на области $\mathbb{R}^n_+$, слагаемые вида
		\[x_j-x_{j,0}-x_{j,0}\ln\dfrac{x_j}{x_{j,0}}\] 
		также неотрицательны в $\mathbb{R}^n_+$.
		Производная $V(x)$ в силу системы:
		\begin{multline*}
			\dot{V}(x)=\sum\limits_{i\in\sigma_1}\hat{k}_i\left(1-\dfrac{x_{i,0}}{x_i}\right)x_i\tilde{f}_i(x)+\sum\limits_{j\in\sigma_2}\hat{k}_j(x_j-x_{j,0})f_j(x)=\\
			=\sum\limits_{i\in\sigma_1}\hat{k}_i(x_i-x_{i,0})\tilde{f}_i(x)+\sum\limits_{j\in\sigma_2}\hat{k}_j(x_j-x_{j,0})f_j(x).
		\end{multline*}
		
		Воспользовавшись методом математической индукции покажем, что ДРФ $f_i(x)$ можно представить как набор произведений разностей $\Delta{}x_j$ и неких дробно-рациональных функций.
		
		При $l=0$ и $m=1$ имеем, что:
		\[f_i(x)=f_i(x)-f_i(x_0)=\dfrac{\Delta{}x_{k_1}}{(x_{k_1}+a_1)(x_{k_1,0}+a_1)}.\]
		
		При $l=0$ и $m=2$:
		\[f_i(x)=f_i(x)-f_i(x_0)=\dfrac{1}{(x_{k_1}+a_1)(x_{k_2}+a_2)}-\dfrac{1}{(x_{k_1,0}+a_1)(x_{k_2,0}+a_2)},\]
		\[f_i(x)=-\dfrac{\Delta{}x_{k_1}}{(x_{k_1}+a_1)(x_{k_1,0}+a_1)(x_{k_2,0}+a_2)}-\dfrac{\Delta{}x_{k_2}}{(x_{k_1}+a_1)(x_{k_2}+a_2)(x_{k_1,0}+a_1)},\]
		
		При $l=0$, $m\ge3$:
		\[f_i(x)=\dfrac{1}{(x_{k_1}+a_1)\cdot...\cdot(x_{k_m}+a_m)}-\dfrac{1}{(x_{k_1,0}+a_1)\cdot...\cdot(x_{k_m,0}+a_m)}.\]
		Тогда приведя дроби к общему знаменателю и совершив в числителе замену $x_{k_j}=\Delta x_{k_j} + x_{k_j,0}$ получим:
		\begin{multline*}
			f_i(x)=-\dfrac{\Delta x_{k_1}(x_{k_2,0}+a_1)\cdot...\cdot(x_{k_m,0}+a_m)}{(x_{k_1}+a_1)\cdot...\cdot(x_{k_m}+a_m)(x_{k_1,0}+a_1)\cdot...\cdot(x_{k_m,0}+a_m)}-\\
			-\dfrac{\Delta x_{k_1}\Delta x_{k_2}(x_{k_3,0}+a_1)\cdot...\cdot(x_{k_m,0}+a_m)}{(x_{k_1}+a_1)\cdot...\cdot(x_{k_m}+a_m)(x_{k_1,0}+a_1)\cdot...\cdot(x_{k_m,0}+a_m)}-...\\
			...-\dfrac{\Delta x_{k_1}\cdot...\cdot\Delta x_{k_{m-1}}(x_{k_m,0}+a_m)}{(x_{k_1}+a_1)\cdot...\cdot(x_{k_m}+a_m)(x_{k_1,0}+a_1)\cdot...\cdot(x_{k_m,0}+a_m)}-\\
			-\dfrac{\Delta x_{k_1}\cdot...\cdot\Delta x_{k_{m}}}{(x_{k_1}+a_1)\cdot...\cdot(x_{k_m}+a_m)(x_{k_1,0}+a_1)\cdot...\cdot(x_{k_m,0}+a_m)}.
		\end{multline*}
		Вынося поочередно множители $\Delta x_{k_j}$ из слагаемых, содержащих их, получим:
		\begin{multline*}
			f_i(x)=-\Delta x_{k_m}\dfrac{(x_{k_1,0}+a_1)\cdot...\cdot(x_{k_{m-1},0}+a_{m-1})}{(x_{k_1}+a_1)\cdot...\cdot(x_{k_m}+a_m)(x_{k_1,0}+a_1)\cdot...\cdot(x_{k_m,0}+a_m)}-\\
			-\Delta x_{k_{m-1}}\dfrac{(x_{k_1,0}+a_1)\cdot...\cdot(x_{k_{m-2},0}+a_{m-2})(\Delta x_{k_m}+x_{k_m,0}+a_m)}{(x_{k_1}+a_1)\cdot...\cdot(x_{k_m}+a_m)(x_{k_1,0}+a_1)\cdot...\cdot(x_{k_m,0}+a_m)}-...\\
			...-\Delta x_{k_1}\dfrac{\Delta x_{k_2}\cdot...\cdot\Delta x_{k_m}+...+(x_{k_2,0}+a_1)\cdot...\cdot(x_{k_m,0}+a_m)}{(x_{k_1}+a_1)\cdot...\cdot(x_{k_m}+a_m)(x_{k_1,0}+a_1)\cdot...\cdot(x_{k_m,0}+a_m)}.
		\end{multline*}
		Заметим, что каждое слагаемое начиная со третьего является произведением дроби 
		\[\frac{\Delta x_{k_j}}{(x_{k_j}+a_j)(x_{k_j,0}+a_j)}\] 
		и одного из предыдущих шагов индукции вплоть до шага $l=0,\, m=m-1$. Таким образом, 
		\begin{multline*}
			f_i(x)=-\dfrac{\Delta x_{k_m}}{(x_{k_1}+a_1)\cdot...\cdot(x_{k_m}+a_m)(x_{k_m,0}+a_m)}-...\\
			...-\dfrac{\Delta x_{k_1}}{(x_{k_1}+a_1)(x_{k_1,0}+a_1)\cdot...\cdot(x_{k_m,0}+a_m)}.
		\end{multline*}
		
		При $l=1$, $m\in\{2,\dots,n\}$:
		\[f_i(x)=\dfrac{(x_{j_1}+b_1)}{(x_{k_1}+a_1)\cdot...\cdot(x_{k_m}+a_m)}-\dfrac{(x_{j_1,0}+b_1)}{(x_{k_1,0}+a_1)\cdot...\cdot(x_{k_m,0}+a_m)}.\]
		Совершив в числителе первого слагаемого замену $x_{j_1}=\Delta x_{j_1} + x_{j_1,0}$ и раскрыв скобки можно получить следующую сумму:
		\begin{multline*}
			f_i(x)=\dfrac{\Delta x_{j_1}}{(x_{k_1}+a_1)\cdot...\cdot(x_{k_m}+a_m)}+\\
			+b_1\dfrac{(x_{k_1,0}+a_1)\cdot...\cdot(x_{k_m,0}+a_m)-(x_{k_1}+a_1)\cdot...\cdot(x_{k_m}+a_m)}{(x_{k_1}+a_1)\cdot...\cdot(x_{k_m}+a_m)(x_{k_1,0}+a_1)\cdot...\cdot(x_{k_m,0}+a_m)}.
		\end{multline*}
		Здесь второе слагаемое соответствует случаю, когда $l=0$ и $m\in\mathbb{N}/\{1\}$, из чего предположение индукции работает также и в этом случае. 
		
		Рассмотрим случай, где $l > 1$ и $m\in\mathbb{N}/\{1\}$.
		\[f_i(x)=\dfrac{(x_{j_1}+b_1)\cdot...\cdot(x_{j_l}+b_l)}{(x_{k_1}+a_1)\cdot...\cdot(x_{k_m}+a_m)}-\dfrac{(x_{j_1,0}+b_1)\cdot...\cdot(x_{j_l,0}+b_l)}{(x_{k_1,0}+a_1)\cdot...\cdot(x_{k_m,0}+a_m)}.\]
		Если разложить произвольное слагаемое в числителе первой дроби как $x_{j_p}=\Delta x_{j_p} + x_{j_p,0},\, p\in\{1,\dots,l\}$, то выражение преобразуется в сумму некой ДРФ и произведения разности $\Delta x_{j_p}$ и некоторой другой ДРФ. При этом заметим, что первое слагаемое будет соответствовать предыдущему шагу индукции, т.е. также может быть приведено к виду произведения разности и ДРФ.
		
		Таким образом, каждая функция $f_i(x)$ может быть представлена как
		\[f_i(x)=\sum\limits_{p\in\hat{\sigma}_i}(x_p-x_{p,0})\tilde{h}_p(x),\]    
		где $\tilde{h}_j(x)$ -- некие ДРФ или линейные функции, $\hat{\sigma}_i$ -- множество всех номеров $x_p,\, p\in\{1,\dots,n\}$, входящих в $f_i(x)$. Тогда производная $V(x)$ в силу системы примет вид:
		\[\dot{V}(x)=\sum\limits_{i\in\sigma_1}\hat{k}_i(x_i-x_{i,0})\sum\limits_{p\in\hat{\sigma}_i}(x_p-x_{p,0})\tilde{h}_{p}(x)+\sum\limits_{j\in\sigma_2}\hat{k}_j(x_j-x_{j,0})\sum\limits_{q\in\hat{\sigma}_j}(x_q-x_{q,0})\tilde{h}_{q}(x).\]
		Сложив все слагаемые с повторяющимися множителями $(x_i-x_{i,0})(x_j-x_{j,0})$ и положив равными нулю коэффициенты при множителях отсутствующих в сумме, получим квадратичную форму с симметричной функциональной матрицей $H$:
		\[\dot{V}(x)=(x-x_0)^{T}H(x-x_0).\qed\]
	\end{proof}
	
	\begin{theorem}
		Система \ref{eq:initial_system} при положительных значениях параметров на границе множества $D$ имеет положения равновесия $P_1\left(0,\, 0,\,0,\,\frac{s_1}{\mu_2}\right)$ и $P_2\left(0,\, c_2,\,0,\,\frac{s_1}{\mu_2}\right)$.  
	\end{theorem}
	
	\begin{theorem}
		$P_1\left(0,\, 0,\,0,\,\frac{s_1}{\mu_2}\right)$ является неустойчивым, а $P_2\left(0,\, c_2,\,0,\,\frac{s_1}{\mu_2}\right)$ асимптотически устойчиво, если 
		\[\dfrac{r_2\mu_2\cap{k}}{c_2}>0,\]
		где $\cap{k}$ -- положительный коэффициент.  
	\end{theorem}
	\begin{proof}
		\qed
	\end{proof}
	
	\begin{theorem}
		Система \ref{eq:initial_system} при значениях параметров, данных в \cite{model}, имеет внутреннее положение равновесия
		\[P_3\left(875419.1750,\,\,943091.7442,\,\,151.6805,\,\,9135.6470,\,\,0.1517\right).\] 
	\end{theorem}
	
	\begin{theorem}
		Положение равновесия $P_3$ является асимптотически устойчивым. 
	\end{theorem}
	 
	\newpage
	\begin{thebibliography}{1}
		\bibitem{model}
		S. Banerjee, S. Khajanchi and S. Chaudhury, PLoS ONE 10(5), e0123611 (2015). 
	\end{thebibliography}
	
\end{document}