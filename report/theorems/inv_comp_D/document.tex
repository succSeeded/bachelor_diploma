% !TeX spellcheck = ru_RU-Russian
% !TeX encoding = UTF-8 

\documentclass[12pt,a4paper]{extarticle}

\usepackage[russian]{babel}
\usepackage[utf8]{inputenc}
\usepackage{setspace} 
\usepackage[a4paper,
	left=30mm,
	right=10mm,
	top=20mm,
	bottom=20mm]{geometry}
\usepackage{amsmath, amssymb}
\usepackage{amsthm}
\usepackage{graphicx, cite}
\usepackage{enumitem}
\usepackage{kprj} 

\VKRTitleOnly

\title{Инвариантные компакты и стабильность положений равновесия в модели взаимодействия клеток иммунитета и мозговой опухоли}
\author{М.Д. Кирдин}
\authorfull{Кирдин Матвей Дмитриевич}
\group{ФН12-81Б}
\faculty{Фундаментальные науки}
\chair{Математическое моделирование}
\chief{А.П. ~Крищенко}
\inspector{М.А. ~Велищанский}
\workyear{2024}
\facultyshort{ФН}
\chairshort{ФН12}
\chairhead{А.П. Крищенко}

\onehalfspacing
\setcounter{secnumdepth}{3}
\renewcommand\qedsymbol{$\blacktriangleright$}
\renewenvironment{proof}{\noindent$\blacktriangleleft$}{}
\theoremstyle{definition}
\newtheorem{theorem}{Теорема}
\theoremstyle{definition}
\newtheorem{definition}{Определение}
\theoremstyle{definition}
\newtheorem{affirmation}{Утверждение}

\begin{document}
	Рассмотрим следующую пятимерную систему с неотрицательными переменными $x=(x_1, x_2, x_3, x_4, x_5)$ и положительными параметрами:
	\begin{equation}
		\begin{cases}
			\begin{aligned}
				\dot{x}_1 &= r_1x_1\left(1-\dfrac{x_1}{c_1}\right)-\dfrac{1}{x_4+e_1}(\alpha_1x_2+\alpha_2x_3)\dfrac{x_1}{x_1+k_1},\\
				\dot{x}_2 &= r_2x_2\left(1-\dfrac{x_2}{c_2}\right)+\dfrac{x_5}{k_4+x_5}a_1\dfrac{1}{x_4+e_2}-\alpha_3\dfrac{x_1}{x_1+k_2}x_2,\\
				\dot{x}_3 &= a_2\dfrac{x_1}{k_5+x_4}-\mu_1x_3-\alpha_4\dfrac{x_1}{x_1+k_3}x_3,\\
				\dot{x}_4 &= s_1 + b_1x_1-\mu_2x_4,\\
				\dot{x}_5 &= b_2x_3-\mu_3x_5,
			\end{aligned}
		\end{cases}\label{eq:initial_system}
	\end{equation}
	где $t\ge0$ --- время;
	
	$x_1$ --- количество клеток глиомы;
	
	$x_2$ --- количество макрофагов;
	
	$x_3$ --- количество т-киллеров;
	
	$x_4$ --- количество белков \textit{TGF-}$\beta$;
	
	$x_5$ --- количество $\gamma$-интерферонов. 
	
	Также из биологических соображений будем полагать, что начальные условия имеют следующий вид:
	\begin{equation}\label{eq:conds}
		x_1(0)\ge0,\,x_2(0)\ge0,\,x_3(0)\ge0,\,x_4(0)\ge0,\,x_5(0)\ge0.
	\end{equation}
	
	Введем следующие обозначения:
	\[\mathbb{R}^n_{+,0}=\{x=(x_1,\dots,x_n)\in\mathbb{R}^n:\, x_i\ge0,\, i=\overline{1,n}\},\,\mathbb{R}_{+,0}=\{x\in\mathbb{R}:\, x\ge0\}.\]
	
	Для системы \ref{eq:initial_system} покажем, что множество $D=\mathbb{R}^{5}_{+,0} = \{x \ge 0\}$ положительно инвариантно, проведем исследование инвариантности пересечений множества $D$ с координатными плоскостями, а также систем, являющихся ограничениями \ref{eq:initial_system} на инвариантные координатные плоскости. Кроме того, найдем компактное множество, содержащее аттрактор системы.
	
	\begin{theorem}
		Множество $D=\mathbb{R}^5_{+,0}$ является положительно инвариантным для системы \ref{eq:initial_system}.
	\end{theorem}
	\begin{proof}
		Заметим, что граница множества $D$ --- множество точек с хотя бы одной нулевой координатой. Таким образом, достаточно показать, для траекторий системы, начинающихся на границе $D$, справедливо, что 
		\[x_i(t)\ge0,\, i=\overline{1,5},\, t\in[0,\,\varepsilon),\,\varepsilon>0.\]
		
		%%%%%%%%%%%%%%%%%%%%%%%%%%%%%%%%%%%%%%%%%%%%%%%%%%%%%%%%%%%%%%%%%%%%%%%%%%%%%%%%%%%%%%%%%%%%%%%%%
		
		
		Рассмотрим случай, когда
		\begin{equation}\label{eq:conds_1}
			x_1(0)=0,\, x_2(0)\ge0,\, x_3(0)\ge0,\, x_4(0)\ge0,\, x_5(0)\ge0.
		\end{equation}
		Для каждого такого начального условия существует $\varepsilon_1>0$ такое, что существует, причем единственное, решение задачи Коши: 
		\[x_1=x_1(t),\, x_2=x_2(t),\, x_3=x_3(t),\, x_4=x_4(t),\, x_5=x_5(t),\, t\in[0,\varepsilon_1),\]
		обращающее систему \ref{eq:initial_system} в тождество. Рассмотрим исходную систему при $t=0$:
		\begin{equation*}
			\begin{cases}
				\begin{aligned}
					\dot{x}_1(0) &= 0,\\
					\dot{x}_2(0) &= r_2x_2(0)\left(1-\dfrac{x_2(0)}{c_2}\right)+\dfrac{x_5(0)}{k_4+x_5(0)}a_1\dfrac{1}{x_4(0)+e_2},\\
					\dot{x}_3(0) &= -\mu_1x_3(0),\\
					\dot{x}_4(0) &= s_1 - \mu_2x_4(0),\\
					\dot{x}_5(0) &= b_2x_3(0) - \mu_3x_5(0).
				\end{aligned}
			\end{cases}
		\end{equation*}
		Решение $x_1(t)\equiv0$ удовлетворяет начальному условию $x_1(0)=0$, а также удовлетворяет первому уравнению исходной системы при $t=0$. При подстановке $x_1(t)\equiv0$ в \ref{eq:initial_system} первое уравнение становится тождеством, а сама система преобразуется к следующему виду:
		\begin{equation*}
			\begin{cases}
				\begin{aligned}
					\dot{x}_2(t) &= r_2x_2(t)\left(1-\dfrac{x_2(t)}{c_2}\right)+\dfrac{x_5(t)}{k_4+x_5(t)}a_1\dfrac{1}{x_4(t)+e_2},\\
					\dot{x}_3(t) &= -\mu_1x_3(t),\\
					\dot{x}_4(t) &= s_1 - \mu_2x_4(t),\\
					\dot{x}_5(t) &= b_2x_3(t)-\mu_3x_5(t).
				\end{aligned}
			\end{cases}
		\end{equation*} 
		Определив $x_2(t),\dots,x_5(t)$ как решения системы с пониженным порядком, из единственности решения задачи Коши имеем, что
		\[x_1\equiv0,\, x_2=x_2(t),\, x_3=x_3(t),\, x_4=x_4(t),\, x_5=x_5(t),\, t\in[0,\,\varepsilon_1),\]
		является решением исходной системы с начальными условиями \ref{eq:conds_1}, лежащим на плоскости $x_1=0$ и не покидающим область $D$ через неё.
		
		%%%%%%%%%%%%%%%%%%%%%%%%%%%%%%%%%%%%%%%%%%%%%%%%%%%%%%%%%%%%%%%%%%%%%%%%%%%%%%%%%%%%%%%%%%%%%%%%%
		
		
		Для каждого из начальных условий вида
		\begin{equation*}
			x_1(0)\ge0,\, x_2(0)\ge0,\, x_3(0)\ge0,\, x_4(0)=0,\, x_5(0)\ge0,
		\end{equation*}
		имеется некое $\varepsilon_2>0$ такое, что существует единственное решение задачи Коши на полуинтервале $t\in[0,\epsilon_2)$, обращающее систему \ref{eq:initial_system} в тождество. В этом случае 
		\[\dot{x}_4(0)=s_1+b_1x_1(0)>0,\]
		т.е. 
		\[x_4(t)>0,\, t\in(0,\, \tilde{\varepsilon}_2),\,\tilde{\varepsilon}_2\le\varepsilon_2\]
		и траектория не выходит из $D$ через плоскость $x_4=0$.
		
		%%%%%%%%%%%%%%%%%%%%%%%%%%%%%%%%%%%%%%%%%%%%%%%%%%%%%%%%%%%%%%%%%%%%%%%%%%%%%%%%%%%%%%%%%%%%%%%%%
		
		
		Рассмотрим случай, когда
		\begin{equation}\label{eq:conds_3}
			x_1(0)\ge0,\, x_2(0)\ge0,\, x_3(0)=0,\, x_4(0)\ge0,\, x_5(0)\ge0.
		\end{equation}
		Для каждого такого начального условия имеется $\varepsilon_3>0$ такое, что существует единственное решение задачи Коши на $t\in[0,\varepsilon_3)$ обращающее систему \ref{eq:initial_system} в тождество. Рассмотрим исходную систему при $t=0$:
		\begin{equation*}
			\begin{cases}
				\begin{aligned}
					\dot{x}_1(0) &= r_1x_1(0)\left(1-\dfrac{x_1(0)}{c_1}\right)-\dfrac{\alpha_1x_2(0)}{x_4(0)+e_1}\dfrac{x_1(0)}{x_1(0)+k_1},\\
					\dot{x}_2(0) &= r_2x_2(0)\left(1-\dfrac{x_2(0)}{c_2}\right)+\dfrac{x_5(0)}{k_4+x_5(0)}a_1\dfrac{1}{x_4(0)+e_2}-\alpha_3\dfrac{x_1(0)}{x_1(0)+k_2}x_2(0),\\
					\dot{x}_3(0) &= a_2\dfrac{x_1(0)}{k_5+x_4(0)},\\
					\dot{x}_4(0) &= s_1 + b_1x_1(0)-\mu_2x_4(0),\\
					\dot{x}_5(0) &= -\mu_3x_5(0).
				\end{aligned}
			\end{cases}
		\end{equation*}
		Если $x_1(0)>0$, то и $\dot{x}_3(0)>0$, из чего $x_1(t)>0,\, t\in(0,\, \tilde{\varepsilon}_3),\,\tilde{\varepsilon}_3\le\varepsilon_3$ и траектория не выходит из $D$ через плоскость $x_3=0$. Если же $x_1(0)=0$, то при $t=0$ система примет вид:
		\begin{equation*}
			\begin{cases}
				\begin{aligned}
					\dot{x}_1(0) &= 0,\\
					\dot{x}_2(0) &= r_2x_2(0)\left(1-\dfrac{x_2(0)}{c_2}\right)+\dfrac{x_5(0)}{k_4+x_5(0)}a_1\dfrac{1}{x_4(0)+e_2},\\
					\dot{x}_3(0) &= 0,\\
					\dot{x}_4(0) &= s_1 + b_1x_1(0)-\mu_2x_4(0),\\
					\dot{x}_5(0) &= -\mu_3x_5(0).
				\end{aligned}
			\end{cases}
		\end{equation*}
		Тогда, аналогично случаю с границей $x_1=0$, $x_1(t)\equiv0$. При этом решение $x_3(t)\equiv0$ удовлетворяет начальному условию $x_3(0)=0$ и уравнению $\dot{x}_3\equiv0$. При его подстановке вместе с $x_1(t)\equiv0$ в исходную систему получим, что:
		\begin{equation*}
			\begin{cases}
				\begin{aligned}
					\dot{x}_2(t) &= r_2x_2(t)\left(1-\dfrac{x_2(t)}{c_2}\right)+\dfrac{x_5(t)}{k_4+x_5(t)}a_1\dfrac{1}{x_4(t)+e_2},\\
					\dot{x}_4(t) &= s_1-\mu_2x_4(t),\\
					\dot{x}_5(t) &= -\mu_3x_5(t).
				\end{aligned}
			\end{cases}
		\end{equation*}
		Если определить $x_2,\, x_4,\, x_5$ как решения системы с пониженным порядком на плоскости $x_1=x_3=0$, то из единственности решения следует, что
		\[x_1\equiv0,\, x_2=x_2(t),\, x_3\equiv0,\, x_4=x_4(t),\, x_5=x_5(t),\, t\in[0,\,\varepsilon_3),\]
		является решением исходной системы для которого выполняются \ref{eq:conds_3} при $x_1(0)=0$, лежащим на плоскости $x_1=x_3=0$ и не покидающим области $D$ через границу $x_3=0$.
		
		%%%%%%%%%%%%%%%%%%%%%%%%%%%%%%%%%%%%%%%%%%%%%%%%%%%%%%%%%%%%%%%%%%%%%%%%%%%%%%%%%%%%%%%%%%%%%%%%%
		
		В случае если 
		\begin{equation}\label{eq:conds_4}
			x_1(0)\ge0,\, x_2(0)\ge0,\, x_3(0)\ge0,\, x_4(0)\ge0,\, x_5(0)=0.
		\end{equation}
		Для каждого такого начального условия существует $\varepsilon_4>0$ такое, что существует единственное решение задачи Коши при $t\in[0,\varepsilon_4)$ обращающее систему \ref{eq:initial_system} в тождество. При $t=0$ исходная система примет вид:
		\begin{equation*}
			\begin{cases}
				\begin{aligned}
					\dot{x}_1(0) &= r_1x_1(0)\left(1-\dfrac{x_1(0)}{c_1}\right)-\dfrac{1}{x_4(0)+e_1}(\alpha_1x_2(0)+\alpha_2x_3(0))\dfrac{x_1(0)}{x_1(0)+k_1},\\
					\dot{x}_2(0) &= r_2x_2(0)\left(1-\dfrac{x_2(0)}{c_2}\right)-\alpha_3\dfrac{x_1(0)}{x_1(0)+k_2}x_2(0),\\
					\dot{x}_3(0) &= a_2\dfrac{x_1(0)}{k_5+x_4(0)}-\mu_1x_3(0)-\alpha_4\dfrac{x_1(0)}{x_1(0)+k_3}x_3(0),\\
					\dot{x}_4(0) &= s_1 + b_1x_1(0)-\mu_2x_4(0),\\
					\dot{x}_5(0) &= b_2x_3(0).
				\end{aligned}
			\end{cases}
		\end{equation*}
		В случае если $x_3(0)>0$ получим, что $\dot{x}_5(0)>0$, из чего $x_5(t)>0,\, t\in(0,\, \tilde{\varepsilon}_4),\,\tilde{\varepsilon}_4\le\varepsilon_4$ и траектория не выходит из $D$ через плоскость $x_5=0$. При $x_3(0)=0$, в свою очередь, система в начальный момент преобразуется к следующему виду:
		\begin{equation*}
			\begin{cases}
				\begin{aligned}
					\dot{x}_1(0) &= r_1x_1(0)\left(1-\dfrac{x_1(0)}{c_1}\right)-\dfrac{\alpha_1x_2(0)}{x_4(0)+e_1}\dfrac{x_1(0)}{x_1(0)+k_1},\\
					\dot{x}_2(0) &= r_2x_2(0)\left(1-\dfrac{x_2(0)}{c_2}\right)-\alpha_3\dfrac{x_1(0)}{x_1(0)+k_2}x_2(0),\\
					\dot{x}_3(0) &= a_2\dfrac{x_1(0)}{k_5+x_4(0)},\\
					\dot{x}_4(0) &= s_1 + b_1x_1(0)-\mu_2x_4(0),\\
					\dot{x}_5(0) &= 0.
				\end{aligned}
			\end{cases}
		\end{equation*} 
		Здесь при $x_1(0)>0$ имеем, что $\dot{x}_3(0)>0$, то есть
		\[\ddot{x}_5(0)=b_2\dot{x}_3(0)-\mu_3\dot{x}_5(0)=b_2\dot{x}_3(0)>0.\] 
		Из этого следует, что $x_5(t)>0,\, t\in(0,\, \tilde{\varepsilon}^\prime_4),\,\tilde{\varepsilon}^\prime_4\le\varepsilon_4$ и траектория не выходит из $D$ через плоскость $x_5=0$. При $x_1(0)=0$ система в начальный момент примет следующий вид:
		\begin{equation*}
			\begin{cases}
				\begin{aligned}
					\dot{x}_1(0) &= 0,\\
					\dot{x}_2(0) &= r_2x_2(0)\left(1-\dfrac{x_2(0)}{c_2}\right),\\
					\dot{x}_3(0) &= 0,\\ 
					\dot{x}_4(0) &= s_1-\mu_2x_4(0),\\
					\dot{x}_5(0) &= 0.
				\end{aligned}
			\end{cases}
		\end{equation*}
		Тогда решения $x_1(t)\equiv0,\, x_3(t)\equiv0,\, x_5(t)\equiv0$ удовлетворяет начальным условиям \ref{eq:conds_4} и уравнениям 
		\[\dot{x}_1(t)=0,\, \dot{x}_3(t)=0,\, \dot{x}_5(0)=0.\] 
		При их подстановке в исходную систему получим, что:
		\begin{equation*}
			\begin{cases}
				\begin{aligned}
					\dot{x}_2(t) &= r_2x_2(t)\left(1-\dfrac{x_2(t)}{c_2}\right),\\
					\dot{x}_4(t) &= s_1-\mu_2x_4(t).\\
				\end{aligned}
			\end{cases}
		\end{equation*}
		Если определить $x_2$ и $x_4$ как решения системы с пониженным порядком на плоскости $x_1=x_3=x_5=0$, то из единственности решения следует, что
		\[x_1\equiv0,\, x_2=x_2(t),\, x_3\equiv0,\, x_4=x_4(t),\, x_5\equiv0,\, t\in[0,\,\varepsilon_4),\]
		является решением исходной системы с начальными условиями \ref{eq:conds_4}, где
		\[x_1(0)=0,\, x_3(0)=0,\, x_5(0)=0,\]
		которое лежит на плоскости $x_1=x_3=x_5=0$ и не покидает область $D$ через границу $x_5=0$.
		
		%%%%%%%%%%%%%%%%%%%%%%%%%%%%%%%%%%%%%%%%%%%%%%%%%%%%%%%%%%%%%%%%%%%%%%%%%%%%%%%%%%%%%%%%%%%%%%%%%
		
		Рассмотрим случай, когда
		\begin{equation}\label{eq:conds_5}
			x_1(0)\ge0,\, x_2(0)=0,\, x_3(0)\ge0,\, x_4(0)\ge0,\, x_5(0)\ge0.
		\end{equation}
		Для каждого такого начального условия также существует $\varepsilon_5>0$ такое, что существует единственное решение задачи Коши на полуинтервале $t\in[0,\,\varepsilon_5)$ обращающее систему исходную систему в тождество. В начальный момент времени \ref{eq:initial_system} принимает вид: 
		\begin{equation*}
			\begin{cases}
				\begin{aligned}
					\dot{x}_1(0) &= r_1x_1(0)\left(1-\dfrac{x_1(0)}{c_1}\right)-\dfrac{\alpha_2x_3(0)}{x_4(0)+e_1}\dfrac{x_1(0)}{x_1(0)+k_1},\\
					\dot{x}_2(0) &= \dfrac{x_5(0)}{k_4+x_5(0)}a_1\dfrac{1}{x_4(0)+e_2},\\
					\dot{x}_3(0) &= a_2\dfrac{x_1(0)}{k_5+x_4(0)}-\mu_1x_3(0)-\alpha_4\dfrac{x_1(0)}{x_1(0)+k_3}x_3(0),\\
					\dot{x}_4(0) &= s_1 + b_1x_1(0)-\mu_2x_4(0),\\
					\dot{x}_5(0) &= b_2x_3(0)-\mu_3x_5(0).
				\end{aligned}
			\end{cases}
		\end{equation*} 
		В случае, если $x_5(0)>0$ получим, что $\dot{x}_2(0)>0$, из чего $x_2(t)>0,\, t\in(0,\, \tilde{\varepsilon}_5),\,\tilde{\varepsilon}_5\le\varepsilon_5$ и траектория не выходит из $D$ через плоскость $x_2=0$. При $x_5(0)=0$, в свою очередь, система преобразуется к следующему виду:
		\begin{equation*}
			\begin{cases}
				\begin{aligned}
					\dot{x}_1(0) &= r_1x_1(0)\left(1-\dfrac{x_1(0)}{c_1}\right)-\dfrac{\alpha_2x_3(0)}{x_4(0)+e_1}\dfrac{x_1(0)}{x_1(0)+k_1},\\
					\dot{x}_2(0) &= 0,\\
					\dot{x}_3(0) &= a_2\dfrac{x_1(0)}{k_5+x_4(0)}-\mu_1x_3(0)-\alpha_4\dfrac{x_1(0)}{x_1(0)+k_3}x_3(0),\\
					\dot{x}_4(0) &= s_1 + b_1x_1(0)-\mu_2x_4(0),\\
					\dot{x}_5(0) &= b_2x_3(0).
				\end{aligned}
			\end{cases}
		\end{equation*}  
		В случае, если $x_3(0)>0$ получим, что $\dot{x}_5(0)>0$, из чего 
		\[\ddot{x}_2(0) = a_1\dfrac{\dot{x}_5(0)}{k_4(x_4(0)+e_2)} > 0.\]
		Тогда $x_2(t)>0,\, t\in(0,\, \tilde{\varepsilon}^{\prime}_5),\,\tilde{\varepsilon}^{\prime}_5\le\varepsilon_5$ и траектория не выходит из $D$ через плоскость $x_2=0$. При $x_3(0)=0$ заметим, что $\ddot{x}_2(0)=0$ и система в начальный момент примет следующий вид:
		\begin{equation*}
			\begin{cases}
				\begin{aligned}
					\dot{x}_1(0) &= r_1x_1(0)\left(1-\dfrac{x_1(0)}{c_1}\right),\\
					\dot{x}_2(0) &= 0,\\
					\dot{x}_3(0) &= a_2\dfrac{x_1(0)}{k_5+x_4(0)},\\
					\dot{x}_4(0) &= s_1 + b_1x_1(0)-\mu_2x_4(0),\\
					\dot{x}_5(0) &= 0.
				\end{aligned}
			\end{cases}
		\end{equation*} 
		Здесь при $x_1(0)>0$ имеем, что $\dot{x}_3(0)>0,\,\ddot{x}_5(0)>0$, то есть $x_3(t)>0,\, t\in(0,\, \tilde{\varepsilon}^{\prime\prime}_5),\,\tilde{\varepsilon}^{\prime\prime}_5\le\varepsilon_5$. Из этого следует, что
		\[\dddot{x}_2(0) = a_1\dfrac{\ddot{x}_5(0)}{k_4(x_4(0)+e_2)}>0.\]
		и тогда $x_2(t)>0,\, t\in(0,\, \tilde{\varepsilon}^{\prime\prime}_5),\,\tilde{\varepsilon}^{\prime\prime}_5\le\varepsilon_5$ и траектория не выходит из $D$ через плоскость $x_2=0$. При $x_1(0)=0$ система в начальный момент примет следующий вид:
		\begin{equation*}
			\begin{cases}
				\begin{aligned}
					\dot{x}_1(0) &= 0,\\
					\dot{x}_2(0) &= 0,\\
					\dot{x}_3(0) &= 0,\\
					\dot{x}_4(0) &= s_1 - \mu_2x_4(0),\\
					\dot{x}_5(0) &= 0.
				\end{aligned}
			\end{cases}
		\end{equation*}
		Тогда решения $x_1(t)\equiv0,\, x_2(t)\equiv0,\, x_3(t)\equiv0,\, x_5(t)\equiv0$ удовлетворяет начальным условиям \ref{eq:conds_4} и уравнениям 
		\[\dot{x}_1(t)=0,\, \dot{x}_2(t)=0,\, \dot{x}_3(t)=0,\, \dot{x}_5(0)=0.\] 
		При их подстановке в исходную систему получим, что:
		\begin{equation*}
			\begin{cases}
				\begin{aligned}
					\dot{x}_4(t) &= s_1-\mu_2x_4(t),\\
				\end{aligned}
			\end{cases}
		\end{equation*} 
		Тогда если определить $x_4$ как решение системы с пониженным порядком на плоскости $x_1=x_2=x_3=x_5=0$, то из единственности решения следует, что
		\[x_1\equiv0,\, x_2\equiv0,\, x_3\equiv0,\, x_4=x_4(t),\, x_5\equiv0,\, t\in[0,\,\varepsilon_5),\]
		является решением исходной системы с начальными условиями \ref{eq:conds_4}, где
		\[x_1(0)=0,\, x_2(0)=0,\, x_3(0)=0,\, x_5(0)=0,\]
		которое лежит на плоскости $x_1=x_2=x_3=x_5=0$ и не покидает область $D$ через границу $x_2=0$. Таким образом. траектории системы не пересекают ни одну из границ $x_i=0,\, i\in\overline{1,5}$.\qed
	\end{proof}
\end{document}