	\section*{Приложение A}
\addcontentsline{toc}{section}{\MakeUppercase{Приложение A}}

\begin{lstlisting}[language=Python, showstringspaces=false, caption=Листинг Файла \texttt{run.py}.]
from model import *
from numpy import set_printoptions
from argparse import ArgumentParser, 
	ArgumentDefaultsHelpFormatter
from math_utils import *
import random
import phaseportrait as phsprt
import numpy as np

set_printoptions(suppress=True, precision=6)
parser = ArgumentParser(formatter_class=
	ArgumentDefaultsHelpFormatter)
parser.add_argument("-d", "--dev", action='store_true', 
	help='Enables development mode. 
		Calculations are done on a less dense net.')
args = vars(parser.parse_args())

model_params = {'r_1': .01, 'c_1': 8.8265e+5, 'e_1': 1e+4,
	'alpha_1': 1.5, 'alpha_2': .12, 'k_1': 2.7e+4, 
	'r_2': .3307, 'c_2': 1e+6, 'a_1': .1163, 
	'k_4': 1.05e+4, 'e_2': 1e+4, 'alpha_3': .0194, 
	'k_2': 2.7e+4, 'a_2': .25, 'k_5': 2e+3, 
	'mu_1': .007, 'alpha_4': .1694, 
	'k_3': 3.3445e+5, 's_1': 6.3305e+4,
	'b_1': 5.75e-6, 'mu_2': 6.93, 'b_2': 1.02e-4,
	'mu_3': .102}

model = eq_model(params=model_params, dev=args['dev'])
model.find_eqpoints()
random.seed()
point = np.array([1.0, 0.0, 0.0, 0.0, 0.0])
model.plot_transitions(point)
model.integrate_at_point(point)
set_printoptions(suppress=True, precision=6)
for i in range(model.eqpoints.shape[0]):
	model.eqpoint_condtitions(model.eqpoints[i])
	print(f'Loss: {model.dxdt(model.eqpoints[i])}.')
	print(f'Jacobian matrix at ({model.eqpoints[i]}):\n
		{model.jacobian(model.eqpoints[i])}')
	print(f'Eigenvalues of J({model.eqpoints[i]}) are:\n
		{model.eigs(model.eqpoints[i])}\n')
bounds = np.array([[0.0, model.Ox1], 
	[0.0, model.Ox2], 
	[0.0, model.Ox3], 
	[model.Ux4, model.Ox4], 
	[0.0, model.Ox5]])
model.integrate_on_set(bounds)
plt.show()
\end{lstlisting}
\newpage
\begin{lstlisting}[language=Python, showstringspaces=false, caption=Файл \texttt{model.py}.]
import numpy as np
import numpy.linalg as LA 
import matplotlib as mpl
import matplotlib.pyplot as plt
import scipy.integrate as scip
import random
from utils import *
from math_utils import *

mpl.rc('text', usetex=True)
mpl.rc('text.latex', preamble=
	r'\usepackage[utf8]{inputenc}')
mpl.rc('text.latex', preamble=
	r'\usepackage[russian]{babel}')
mpl.rcParams.update({'font.size': 14})
	
class eq_model:
	
	def __init__(self, params=None, dev=None):
		self.__params_set = False
		if dev:
			self.dev = True
		else:
			self.dev = False
		if params and isinstance(params, dict):
			self.__set_parameters(params)
		
	def __set_parameters(self, params=None):
		if params and isinstance(params, dict):
			print('Setting parameters...')
			for param in [*params.items()]:
				self.__setattr__(param[0], param[1])
			self.__set_variables_at_eqpoints()
			self.__set_dequations()
			self.__set_jacobian()
			self.__set_attr_bounds()
			self.__params_set = True
			print('Parameters set!')
		else:
			raise ValueError('System parameters should be 
				a non-empty dict')
		
	def __set_variables_at_eqpoints(self):
		self.__setattr__('x_4', lambda x_1: 
			(self.b_1*x_1 + self.s_1)/self.mu_2)
		self.__setattr__('x_3', lambda x_1: 
		self.a_2*x_1*(x_1+self.k_3)/
			((self.k_5+self.x_4(x_1))*
			(self.alpha_4*x_1+self.mu_1*(x_1+self.k_3))))
		self.__setattr__('x_5', lambda x_1: 
			self.b_2*self.x_3(x_1)/self.mu_3)
		self.__setattr__('x_2', lambda x_1: 
			(self.r_1*(1.0-x_1/self.c_1)*
			(self.x_4(x_1)+self.e_1)*(x_1+self.k_1)-
			self.alpha_2*self.x_3(x_1))/self.alpha_1)
		
	def __target_function(self, x_1):
		return self.r_2*self.x_2(x_1)*(self.c_2-self.x_2(x_1))*
			(x_1+self.k_1)*(self.k_4+self.x_5(x_1))*
			(self.x_4(x_1)+self.e_2)+self.a_1*self.c_2*
			(x_1+self.k_1)-self.alpha_3*self.c_2*x_1*
			self.x_2(x_1)*(self.k_4+self.x_5(x_1))*
			(self.x_4(x_1)+self.e_2)
		
	def __set_dequations(self):
		def dxdt(X):
			try:
				if X.shape[0] != 5:
					raise ValueError('Incorrect argument 
						dimensions.')
			except:
				if isinstance(X,list) and 
					isinstance(X[0],np.array):
				if X[0].shape != 5:
					raise ValueError('Incorrect argument 
						dimensions.')
				else:
					raise TypeError('Input is neither list of 
						np.array nor an np.array.')
			
			return np.array(
				[self.r_1*X[0]*(1.0-X[0]/self.c_1)-
					(self.alpha_1*X[1]+self.alpha_2*X[2])
					*X[0]/(X[3]+self.e_1)/(X[0]+self.k_1), 
				self.r_2*X[1]*(1-X[1]/self.c_2)+self.a_1*
					X[4]/((self.k_4+X[4])*(self.e_2+X[3]))-
					self.alpha_3*X[0]*X[1]/(X[0]+self.k_2), 
				self.a_2*X[0]/(X[3]+self.k_5)-self.mu_1*X[2]
					-self.alpha_4*X[0]*X[2]/(X[0]+self.k_3), 
				self.s_1+self.b_1*X[0]-self.mu_2*X[3], 
				self.b_2*X[2]-self.mu_3*X[4]], dtype='float64')      
		
		self.__setattr__('dxdt', dxdt)
		
	def __set_jacobian(self):
		self.__setattr__('jacobian', lambda X: 
			np.transpose(np.array([partial(self.dxdt, X, 
			n_arg=j) for j in range(1,6)], dtype='float64')))
		
	def __set_attr_bounds(self):
		self.__setattr__('Ox1', self.c_1)
		self.__setattr__('Ox4', self.s_1/self.mu_2 
			+ self.b_1*self.Ox1)
		self.__setattr__('Ux4', self.s_1/self.mu_2)
		self.__setattr__('Ox3', self.a_2*self.Ox1*(self.Ox1 +
			self.k_2)/(self.k_5+self.Ux4)/self.mu_1/self.k_2)
		self.__setattr__('Ox5', self.b_2*self.Ox3/self.mu_3)
		self.__setattr__('Ox2', self.c_2*.5 + 
			np.sqrt(self.c_2*self.c_2*.5 + self.a_1*self.Ox5/
			self.k_4/(self.Ux4+self.e_2)))
		
		print(f'\nAttractor-containing set boundaries 
			(Oxi -- upper bound, Uxi -- lower bound):\n
			\nOx1 = {self.Ox1}')
		print(f'Ox2 = {self.Ox2}')
		print(f'Ox3 = {self.Ox3}')
		print(f'Ux4 = {self.Ux4}')
		print(f'Ox4 = {self.Ox4}')
		print(f'Ox5 = {self.Ox5}\n')
		
	def eigs(self, point):
		return np.sort(LA.eigvals(self.jacobian(point)))
		
	def find_eqpoints(self):
		self.eqpoints = np.vstack((
			np.array([0.0,0.0,0.0,self.s_1/self.mu_2,0.0]),
			np.array([0.0,self.c_2,0.0,self.s_1/self.mu_2,0.0])
			))
		
		print(f'Searching for zeros...')
		zeros = []
		
		if self.dev == True:
			interval = (self.c_1-100000, self.c_1)
		else:
			interval = (1e-6, self.c_1)
		
		start = time.time()    
		zeros, zero_times = zero_localizer(
		self.__target_function, interval, d=1e-1)
		end = time.time()
		for x_1 in zeros:
			self.eqpoints = np.vstack((self.eqpoints,
				np.array([x_1, self.x_2(x_1), self.x_3(x_1), 
				self.x_4(x_1), self.x_5(x_1)])))
			print(f'Total time elapsed: {round(end-start, 6)}s')
			print(f'Time elapsed while find_zero: 
				{[f"{round(time,6)}s" for time in zero_times]}')
		
		if len(zeros) == 1:
			print(f'One zero found. x_1 = {zeros[0]}')
		elif len(zeros) > 0:
			print(f'{len(zeros)} zeros found.')
			[print(f'X = {zero}') for zero in zeros]
		else:
			print('No zeros found on a given interval.')
		
		print('\nEquilibrium points:\n')
		for i in range(self.eqpoints.shape[0]):
			print(f'P = {self.eqpoints[i,:]}\n')
		
		return None
		
	def eqpoint_condtitions(self, point):
		cond_1 = self.r_1*(self.c_1-point[0])*
			(point[3]+self.e_1)*(point[0]+self.k_1) -
			self.alpha_2*self.c_1*point[2] > 0
		cond_2 = (0 < point[0]) and (point[0] < self.c_1)
		if cond_1 and cond_2:
			print(f'Point X = {point} satisifies condtitions 
				for a equilibrium point.')
		else:
		print(f'Point X = {point} does not satisfy 
			condtitions for a equilibrium point.')
		return cond_1 and cond_2
		
	def cond_roots(self):
		d = np.array([-self.c_1, (self.s_1+self.mu_2*self.e_1)/
			self.b_1, self.k_1, (self.s_1+self.mu_2*self.k_5)/
			self.b_1, self.mu_1*self.k_3/(self.alpha_4+
			self.mu_1)])
		f = self.alpha_2*self.c_1*self.a_2*self.mu_2*
			self.mu_2/self.r_1/self.b_1/self.b_1/
			(self.alpha_4+self.mu_1)
		P = 1
		for i in range(5):
			P = np.polymul(P,(1, d[i]))
		P = np.polyadd(P, [f, f*self.k_3, 0])
		croots = np.roots(P)
		print(f'Коэффициенты многочлена из (4): {P}')
		print(f'{np.polyval(P, np.roots(P))}')
		for j in range(croots.shape[0]):
			point = [croots[j], self.x_2(croots[j]), 
				self.x_3(croots[j]), self.x_4(croots[j]), 
				self.x_5(croots[j])]
			val = self.r_1*(self.c_1-point[0])*(point[3]+
				self.e_1)*(point[0]+self.k_1) - self.alpha_2*
				self.c_1*point[2]
			print(f'(4)-2 в точке x_1 = {croots[j]}: {val}')
		return croots
		
	def integrate_at_point(self, point, T = 2000.0):
		
		ax = [plt.figure().add_subplot(projection='3d') 
			for i in range(3)]
		sol = scip.solve_ivp(lambda t, X: self.dxdt(X), 
			[0.0,T], point, rtol=1e-7, atol=1e-6)
		X = sol.y
		for i in range(len(ax)):
			ax[i].plot(X[(i + 2*(i//5))%5,:], X[(i+1)%5,:],
				X[(i+2-2*(i//5))%5,:], color='black')    
			ax[i].scatter(point[(i + 2*(i//5))%5], 
				point[(i+1)%5], point[(i+2-2*(i//5))%5], 
				color='black')    
			ax[i].scatter(self.eqpoints[1:,(i + 2*(i//5))%5],
				self.eqpoints[1:,(i+1)%5], self.eqpoints[1:,
				(i+2-2*(i//5))%5], color='r')
			ax[i].text(self.eqpoints[1,(i + 2*(i//5))%5], 
				self.eqpoints[1,(i+1)%5], self.eqpoints[1,
				(i+2-2*(i//5))%5], '$P_2$')
			ax[i].text(self.eqpoints[2,(i + 2*(i//5))%5], 
				self.eqpoints[2,(i+1)%5], self.eqpoints[2,
				(i+2-2*(i//5))%5], '$P_3$')
			ax[i].set_xlabel(f'$x_{(i + 2*(i//5))%5+1}$')
			ax[i].set_ylabel(f'$x_{(i+1)%5+1}$')
			ax[i].set_zlabel(f'$x_{(i+2-2*(i//5))%5+1}$')
		return None 
		
	def quiver(self, plot_area, N = np.array([5, 5, 5, 5, 5])):
		
		ax = plt.figure().add_subplot(projection='3d')
		
		# Make the grid
		x1, x2, x3, x4, x5 = np.meshgrid(np.linspace(
			plot_area[0,0], plot_area[0,1], num = N[0]),
		np.linspace(plot_area[1,0], plot_area[1,1],
			num = N[1]),
		np.linspace(plot_area[2,0], plot_area[2,1],
			num = N[2]),
		np.linspace(plot_area[3,0], plot_area[3,1],
			num = N[3]),
		np.linspace(plot_area[4,0], plot_area[4,1],
			num = N[4]))
		
		# Make the direction data for the arrows
		u1, u2, u3, u4, u5 = self.dxdt(np.array([x1,
			x2, x3, x4, x5])) 
		
		ax.quiver(x1, x2, x3, u1, u2, u3, length = 1)
		ax.scatter(self.eqpoints[2,0], self.eqpoints[2,1], 
			self.eqpoints[2,2], color='r')
		return None
		
	def integrate_on_set(self, bounds, T = 2000.0, 
		N = np.array([5,5,5])):
		
	# working under assumption that the first element of 
	# self.eqpoints is an equilibrium point that lies 
	# inside of the set D 
		plt.rcParams['text.usetex'] = True
		
		if bounds.shape != (5,2):
			raise ValueError('Incorrect bounds!')
		
		x1 = np.linspace(bounds[0,0], bounds[0,1], 
			num = N[0])
		x3 = np.linspace(bounds[2,0], bounds[2,1], 
			num = N[1])
		x5 = np.linspace(bounds[4,0], bounds[4,1], 
			num = N[2])
		ax = plt.figure().add_subplot(projection='3d')
		ax.set_prop_cycle(color=mpl.cm.viridis(
		np.linspace(0,1,N[0]*N[1]*N[2])))
		points = np.array([np.array([x1[i], 0.0, x3[j], 
			self.s_1/self.mu_2, x5[k]]) 
			for i in range(len(x1)) 
			for j in range(len(x3))  
			for k in range(len(x3))])
		for i in range(len(points)):
			sol = scip.solve_ivp(lambda t, X: self.dxdt(X), 
				[0.0,T], points[i], rtol=1e-7, atol=1e-6)
			X = sol.y
			ax.plot(X[0,:], X[2,:], X[4,:])    
			ax.scatter(self.eqpoints[1:,0], self.eqpoints[1:,2], 
			self.eqpoints[1:,4], color='r')
			ax.text(self.eqpoints[1,0], self.eqpoints[1,2], 
			self.eqpoints[1,4], '$P_2$')
			ax.text(self.eqpoints[2,0], self.eqpoints[2,2], 
			self.eqpoints[2,4], '$P_3$')
			ax.set_xlabel(f'$x_{1}$')
			ax.set_ylabel(f'$x_{3}$')
			ax.set_zlabel(f'$x_{5}$')
		return None
		
	def plot_transitions(self, point, T = 3000.0):
		
		ax1 = [plt.figure().add_subplot() for i in range(3)]
		ax2 = [plt.figure().add_subplot() for i in range(2)]
		sol = scip.solve_ivp(lambda t, X: self.dxdt(X), 
			[0.0,T], point, rtol=1e-7, atol=1e-6)
		for i in range(len(ax1)):
			ax1[i].grid()
			if i == 0 or i == 3:
				ax1[i].plot(sol.t, sol.y[i,:])
			else:
				ax1[i].plot(sol.t[:500], sol.y[1,:500])
				ax1[i].set_xlabel('t, дней')
				ax1[i].set_ylabel(f'$x_{i+1}$')
		random.seed()
		init_point = np.array([0.0, random.random()*self.Ox2,
		0.0, random.random()*self.Ox4, 0.0])
		sol = scip.solve_ivp(lambda t, X: self.dxdt(X), 
			[0.0,T], init_point, rtol=1e-7, atol=1e-6)
		for i in range(len(ax2)):
			ax2[i].grid()
			ax2[i].set_xlabel('t, дней')
			ax2[i].set_ylabel(f'$x_{(i+1)*2}$')
		ax2[0].plot(sol.t[:100], sol.y[1,:100])
		ax2[1].plot(sol.t[:30], sol.y[3,:30])
		return None
\end{lstlisting}
\newpage

\begin{lstlisting}[language=Python, showstringspaces=false, caption=Файл \texttt{math\_utils.py}.]
import numpy as np
import time
from utils import progress_bar
	
def der1(func, point, h=1e-9):
	return (func(point+h*0.5)-func(point-h*0.5))/h
	
def partial(func, point, n_arg=1, h=1e-9):
	step = np.zeros(point.shape[0])
	step[n_arg-1] = h
	return (func(point+step*0.5) - 
		func(point-step*0.5))/h
	
def der2(func, point, h=1e-9):
	return (func(point+h) - 2.0*func(h) + 
		func(point-h))/(h**2)
	
def find_zero(func, interval, method='linear', j=0, 
	time_this=False):   
	
	if time_this:
		start = time.time()
	
	match method.lower():
	case 'linear':
		zero = interval[0]-(interval[1]-interval[0])*
			loss/(loss_next-loss)
	case 'halley':
		zero = (interval[1]+interval[0])*0.5
		i = j
		while True:
			i += 1
			zero_prev = zero 
			zero = zero - func(zero)/(der1(func, zero)
			- 0.5*func(zero)*der2(func,zero)/
			der1(func,zero))
			if np.abs(func(zero)) < 1e-6 or i > 1e+5:
				break
	case 'golden_section':
	
		a = interval[0]
		b = interval[1] 
		eps = 1e-7
		t = (1 + np.sqrt(5))*0.5
		x1 = a + (1 - 1/t) * (b - a)
		x2 = a + 1/t * (b - a)
		l = b - a
		f1 = f(x1) #inital function values
		f2 = f(x2)
		
		while (l > eps): 
			if (f1 > f2):
				a = x1
				x1 = x2
				f1 = f2
				x2 = b - (b - a)/(t+1)
				f2 = f(x2)
			else:
				b = x2
				x2 = x1
				f2 = f1
				x2 = a + (b - a)/(t+1)
				f1 = f(x1)
				l = b - a
				x = (a + b)*0.5
				zero = x
	
	if time_this:
		end = time.time()
		return (zero, end-start)
	
	return (zero, None)
	
def zero_localizer(func, interval, d=1e-1, k=0):
""" 
@param func (callable): Function for which zeros 
	are localized. 

@param interval (ArrayLike): Interval in which zeros 
	are localized. 

@param d (float): Diameter of interval division.

@param k (int): number of recursive function calls. 

This function finds zeros by continuously searching 
	for sign changes on nodes of a division of interval 
	and then subdivide it until a zero is found.
"""
	zeros = []
	zero_times = []
	points = np.arange(interval[0], interval[1], d)
	target = func(points)
	
	for i in range(len(points)-1):
		if k == 0 and i%1e+5==0:
			progress_bar(i, len(points)-2)
		
		if (target[i]*target[i+1] < 0) and 
			(np.abs(d) > 1e-6):
			zeros, zero_times = zero_localizer(func, 
				points[i:i+2], d=d*1e-1, k=k+1)
		elif (target[i]*target[i+1] < 0) and 
			(np.abs(d) <= 1e-6):
			zero, zero_time = find_zero(func, 
			points[i:i+2], method='halley', 
			time_this=True)
			zeros += [zero]
		zero_times += [zero_time]
		
		if k == 0:
			progress_bar(i, len(points)-2)
	
	return (zeros, zero_times)
\end{lstlisting}